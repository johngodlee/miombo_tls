\documentclass[11pt,a4paper]{article}

% Define page geometry
\usepackage{geometry} \geometry{left=2.2cm, right=2.2cm, top=2.2cm, bottom=2cm}
\parskip 0.15cm 
\setlength{\parindent}{0cm} 
\usepackage{pdflscape}
\usepackage[document]{ragged2e}

% Text formatting
\usepackage[T1]{fontenc}  % Set font

\usepackage{lineno}  % Line numbers

\usepackage{amssymb}  % Symbols
\usepackage{textcomp}
\newcommand{\textapprox}{\raisebox{0.5ex}{\texttildelow}}  % Command for a good tilde

\linespread{1.5}  % Linespacing

\usepackage{xcolor} \newcommand{\TODO}[1]{\textcolor{red}{\textbf{#1}}}   %

\usepackage{lineno}

% Tables
\usepackage{multirow} \setlength{\tabcolsep}{4pt}

% Image handling
\usepackage{graphicx} 

\makeatletter \g@addto@macro\@floatboxreset\centering  
\makeatother

\graphicspath{ {img/} }  % Define image path

\usepackage{subfig}  % Compound figures

\usepackage{float}  % Precise figure location

% Bibliography management
\usepackage[style=authoryear, natbib=true, backend=biber]{biblatex}
\addbibresource{lidar.bib}

% Links within document, nice figure formatting
\usepackage[breaklinks]{hyperref} \definecolor{links}{RGB}{0,0,0} \hypersetup{
	breaklinks, colorlinks=true, linkcolor=links, anchorcolor=links,
	citecolor=links, filecolor=links, menucolor=links, runcolor=links,
	urlcolor=links, pdfauthor={John L. Godlee} }
	\def\subsectionautorefname{section} \def\subsubsectionautorefname{section}

\newcommand{\beginsupplement}{% 
	\setcounter{table}{0}
	\renewcommand{\thetable}{S\arabic{table}}% 
	\setcounter{figure}{0}
	\renewcommand{\thefigure}{S\arabic{figure}}% 
}
     
% Variables
\newcommand{\rawpt}{2.9e+08}
\newcommand{\voxelpt}{4.5e+07}
\newcommand{\subpt}{2.1e+07}


\newcommand{\titletext}{Terrestrial laser scanning}

\begin{document}

{\LARGE{\titletext{}}}

\linenumbers

\section*{Abstract}

\section{Introduction}

The characterization of tree canopy structure in wooded ecosystems constitutes a long-standing field of research that has been fundamental to interpreting, modelling, and improving understanding of ecosystem function \citep{Watt1947, Whittaker1969, Horn1971, Maarel1996}. Canopy structure describes the spatial distribution and density of canopy foliage, comprising the primary interface between trees, the atmosphere and sunlight. It is therefore essential to understand the drivers of variation in canopy structure to improve modelling of earth-atmosphere carbon fluxes and community assembly \citep{}. 

At continental scales, variation in canopy height and canopy cover, two coarse measures of canopy structure both of which have been shown to affect woody productivity and correlate with woody biomass \citep{}, can largely be explained by climate and edaphic data \citep{SOME-GEDI}. Increased resource availability allows for larger trees and more closed canopies \citep{}. At the scale of a single tree community however, where variation in climate and soil may be negligible, variation in canopy structure is thought to be affected principally by an interacting combination of tree canopy species composition \citep{}, and disturbance history \citep{}. However, empirical testing of these mechanisms thought to drive canopy structure in natural wooded ecosystems remains sparse across many biomes \citep{}.

Following established biodiversity-ecosystem function theory, the niche partitioning of canopy space, i.e. the spatial complementarity of individual tree canopies, hereafter referred to as `crown complementarity', is thought to be a key mechanism underlying positive biodiversity-productivity effects in wooded ecosystems \citep{Pretzsch2014, Barry2019}. Biodiversity-ecosystem function theory predicts that crown complementarity and thus canopy complexity and foliage density should increase with tree diversity in the local neighbourhood, thus increasing standing biomass and woody productivity, as coexisting species must occupy non-identical niche space to avoid competitive exclusion \citep{Gadow1993}. 

As well as the species diversity of trees in a local neighbourhood, the spatial distribution and relative size dominance of those trees, i.e. stand structure, is also expected to affect crown complementarity and canopy structure. Increased heterogeneity in tree size, whether a result of species diversity or disturbance history, is expected to increase crown complementarity as individuals of different sizes can occupy different layers of the canopy \citep{}. Additionally, clustering of individuals in space is expected to increase canopy structural heterogeneity across a stand, but ultimately decrease total foliage density due to an increase in competitive interactions \citep{}. Clustering may occur as a result of disturbance history, or as a result of strong facilitation effects among individuals in a hostile environment \citep{Ratcliffe2017}.

While much work in the field of forest management has been done to test biotic drivers of tree canopy structure in temperate \citep{} and boreal forests \citep{}, similar work in the tropics is comparatively scarce \citep{}. In dry tropical woodlands especially, tree canopy structure and its effect on ecosystem function has received little attention, possibly due to the misplaced assumption that woody productivity in these ecosystems does not represent a globally significant carbon flux \citep{}, or that tree canopies in these smaller stature woodlands do not interact and compete for resources to the same degree as in large stature forests \citep{}. In recent years however, it has been shown that dry tropical woodlands represent the largest uncertainty in our estimates of the terrestrial carbon cycle \citep{Quere2018, Ahlstrom2015}. \citet{Sitch2015} demonstrated the dominant role of the dry tropics in driving variability in the terrestrial carbon sink, and showed that the dry tropics are the fastest increasing component of the terrestrial carbon sink. Part of this uncertainty arises from our lacking a nuanced understanding of how species composition and structure affect ecosystem function in these ecosystems, which underpins the Dynamic Global Vegetation Models (DGVMs) fed into global carbon dynamics models. This knowledge gap prompts further research of the biotic drivers of variation in productivity in the dry tropics, of which canopy structure is a key part \citep{}.

Canopy structure is multi-dimensional and has previously been explained using a plethora of simple metrics that originated in forest and community ecology \citep{}. Assessments of canopy structure in the dry tropical have most often modelled tree canopies as a series of ellipses (2D) or ellipsoids (3D) based on field measurements with measuring tapes \citep{}. Measurements of this kind are time consuming and yet are an over-simplification of canopy structure \citep{}. Alternatively, canopy cover is often measured using indirect optical methods which partition sky from canopy material, i.e. with hemispherical photography or the commonly used LAI-2000, providing a 2D representation of the canopy but lacking information on vertical canopy structure. In recent years, particularly in temperate and boreal forests, LiDAR (Light Detection And Ranging) has emerged as a suitable technology for rapidly and precisely assessing canopy structure in 3D, conserving information on 3D structure of the calibre that is required to understand it's complexities \citep{}.

In this study we applied terrestrial LiDAR techniques to woodland-savanna mosaics at two sites in southern Africa, with the aim of increasing our understanding of how various metrics of tree canopy structure are affected by tree neighbourhood diversity and stand structure. Our overarching contention is that neighbourhoods of greater tree diversity and greater structural diversity allow greater canopy complexity and foliage density, resulting in higher productivity, and ultimately a more `forest-like' community, rather than an open canopy savanna.

\section{Materials and methods}

\subsection{Study sites}

Measurements were conducted at two sites, the first in Bicuar National Park, southwest Angola (S15.1$^\circ$, E14.8$^\circ$), and the second in and around Mtarure Forest Reserve, southeast Tanzania (S9.0$^\circ$, E39.0$^\circ$) (\autoref{map}). At each site, 1 ha plots were sited in areas of miombo woodland vegetation, across a gradient of stem density. In Angola, 15 plots were sampled, while in Tanzania, seven were sampled following the curtailment of fieldwork due to COVID-19 travel restrictions. Fieldwork was conducted between February and April at both sites, during the peak growth period of each site in order to capture the highest foliage volume in the canopy.

\begin{figure}[H]
\centering
	\includegraphics[width=\textwidth]{map}
	\caption{Location of study sites within southern Africa (a), and of 1 ha plots within each site. The blue polygons denote the boundaries of protected areas which encompass the majority of study sites, Bicuar National Park in Angola (b), and Mtarure Forest Reserve in Tanzania (c). The background of each site map is a re-classified version of the GlobCover global land cover classification \citep{Globcover}.}
	\label{map}
\end{figure}

\subsection{Field measurements}

Each plot was further subdivided into nine 10 m diameter circular subplots arranged in a regular grid, with a buffer from the plot edge (\autoref{subplot}). For each subplot, we measured all woody stems >5 cm trunk diameter with canopy material inside the subplot. We identified each stem to species and measured trunk diameter (diameter at breast height - 1.3 m), height to top of canopy material, canopy area calculated as an ellipse of two perpendicular crown diameter measurements, distance and direction of stem from the subplot centre.

At the centre of each subplot a photograph was taken with a Nikon D750 full-frame DSLR camera, with a Sigma 8 mm f/3.5 EX DG circular fisheye lens. The lens has an equisolid (equal area) projection, which avoids image distortion. Photos were taken facing directly to zenith, with the top of the camera facing to magnetic north, at a height of 1.3 m or above understorey vegetation, whichever was higher. Photos were captured under uniform light conditions as much as possible, either under overcast skies or early in the day before direct sunlight could be seen on the photo \citep{MacFarlane2011}. 

\begin{figure}[H]
\centering
	\includegraphics[width=\textwidth]{subplot}
	\caption{The layout of 10 m diameter subplots within each 1 ha square plot (left) and the layout of a single subplot (right). Each subplot is situated inside a 15 m buffer from the plot edge, with 35 m between subplot centres. Subplots are arranged in a 3x3 grid. Disc-pasture measurements and biomass samples are located in cardinal directions 2 m from the centre of the subplot. All distances are in metres.}
	\label{subplot}
\end{figure}

\subsection{Terrestrial laser scanning}

Within each subplot, a variable number of scans were recorded using a Leica HDS6100 phase-shift terrestrial laser scanner (TLS) \citep{Leica}. The number and position of scans within a subplot was determined by the arrangement of canopy material in the subplot. Scan positions were arranged to minimise shadows within the canopy, and to maximise canopy penetration. The number of scans per subplot ranged between one and five in both Angola and Tanzania. Registration of multiple scans from different locations around each subplot minimised the occlusion effect and improved canopy penetration.

\subsection{Data analysis}

\subsubsection{Scan processing}

Point clouds from scans in each subplot were registered and unified using Leica Cyclone (version 9.1). Targets from each scan were aligned using Cyclone's automatic target acquisition. 

Point clouds were voxelised to cubic voxel sizes of different sizes depending on the application of the data. For subplot height profile estimation and gap fraction we used 5 cm\textsuperscript{3} voxels, and for whole plot canopy rugosity we used 10 cm\textsuperscript{3} voxels. Variation in voxel size reflects the spatial scale of each analysis, and is bounded by the beam divergence of the scanner \citep{}. Choosing voxels that are too small can result in pock-marked representations of surfaces that are especially problematic when estimating canopy structure at a larger scale, such as when estimating canopy top roughness, while voxels that are too large can result in an over-estimation of plant volume when estimating canopy foliage density \citep{Cifuentes2014}. Voxels were classed as filled if they intersected with one or more points.

Partial object interceptions caused by phase-shift laser scanners can produce erroneous results and must be corrected for to accurately estimate canopy height \citep{}. We used a noise reduction algorithm from \citet{} to discard points that appeared far from other points. This effectively removed ghost points produced by partial interceptions and also removed many erroneous returns caused by airborne dust particles, which was common in our study site.

Ground points were classified using the Progressive Morphological Filter (PMF) from \citet{Zhang2003}. Point cloud height was reclassified height based on this revised ground layer by measuring the vertical distance between the nearest ground point and each point.

Raw points clouds for each subplot had \textapprox{}\rawpt{} points, \textapprox{}\voxelpt{} points after voxelisation, and \textapprox{}\subpt{} points after noise reduction.

We used ray-tracing to calculate gap fraction from TLS scans at the centre of each subplot. Hemispherical images were created using the POV-ray software \citep{}. Voxels were converted to matt black cubes filling the voxel volume, with a white sky box and no light source. A `camera' with a 180\textdegree{} fisheye lens was placed at the subplot centre at a height of 1.8 m pointin directly upwards. The images produced by POV-ray were analysed using Hemiphot \citep{Steege} to estimate canopy gap fraction.

\subsection{Stand structure}

For each subplot, we calculated an adapted version of the Hegyi index to estimate crowding \citep{Hegyi1974}.

At the plot level, we estimated the regularity of species distribution using the spatial mingling index \citep{Gadow}. We also measured whole plot stand structure using the Winkelmass \citep{}.

\subsection{Statistical analysis}

Linear mixed effects models tested the effects of diversity and stand structural metrics on canopy structure.

\section{Results}

\subsection{Vertical canopy complexity}

The linear mixed effects models showed that species richness of the subplot neighbourhood had variable effects across the measures of canopy structure (\autoref{height_profile_mods_fe}), but the effect sizes were not significant for any model (\autoref{height_profile_mod_rich_slopes}). On the other hand, stand physical structure had a much greater effect on canopy structure variables. The Hegyi index and coefficient of variation of stem diameter had positive significant effects on foliage density (AUC gap fraction), effective number of layers, and canopy max height.

\begin{figure}[H]
\centering
	\includegraphics[width=\textwidth]{height_profile_mod_rich_slopes}
	\caption{Standardized fixed effect slopes for each model of a canopy structure metric. Slope estimates are $\pm$1 standard error. Slope estimates where the interval (standard error) does not overlap zero are considered to be significant effects.}
	\label{height_profile_mod_rich_slopes}
\end{figure}

\subsection{Canopy rugosity}

None of the stand structure metrics or diversity metrics had significant effects on whole-plot canopy rugosity, but the model as a whole had a significant effect on canopy rugosity

\begin{figure}[H]
\centering
	\includegraphics[width=0.4\textwidth]{rugosity_mod_slopes}
	\caption{Standardized fixed effect slopes for whole-plot canopy rugosity. Slope estimates are $\pm$1 standard error. Slope estimates where the interval (standard error) does not overlap zero are considered to be significant effects.}
	\label{rugosity_mod_slopes}
\end{figure}

\section{Discussion}

Species diversity didn't have strong effects on canopy structure, but stand structure did.

\subsection{Scaling up from subplots}

\section{Conclusion}

\printbibliography

%\section{Supplementary Material} \beginsupplement

\end{document}
