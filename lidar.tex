\documentclass[11pt,a4paper]{article}

% Define page geometry
\usepackage{geometry} \geometry{left=2.2cm, right=2.2cm, top=2.2cm, bottom=2cm}
\parskip 0.15cm \setlength{\parindent}{0cm} \usepackage{pdflscape}
\usepackage[document]{ragged2e}

% Text formatting
\usepackage[T1]{fontenc}  % Set font

\usepackage{lineno}  % Line numbers

\usepackage{amssymb}  % Symbols

\linespread{1.25}  % Linespacing

\usepackage{xcolor} \newcommand{\todo}[1]{\textcolor{red}{\textbf{#1}}}   %

% Tables
\usepackage{multirow} \setlength{\tabcolsep}{4pt}

% Image handling
\usepackage{graphicx} 

\makeatletter \g@addto@macro\@floatboxreset\centering  
\makeatother

\graphicspath{ {img/} }  % Define image path

\usepackage{subfig}  % Compound figures

\usepackage{float}  % Precise figure location

% Bibliography management
\usepackage[style=authoryear, natbib=true, backend=biber]{biblatex}
\addbibresource{lidar.bib}

% Links within document, nice figure formatting
\usepackage[breaklinks]{hyperref} \definecolor{links}{RGB}{0,0,0} \hypersetup{
	breaklinks, colorlinks=true, linkcolor=links, anchorcolor=links,
	citecolor=links, filecolor=links, menucolor=links, runcolor=links,
	urlcolor=links, pdfauthor={John L. Godlee} }
	\def\subsectionautorefname{section} \def\subsubsectionautorefname{section}

\newcommand{\beginsupplement}{% 
	\setcounter{table}{0}
	\renewcommand{\thetable}{S\arabic{table}}% 
	\setcounter{figure}{0}
	\renewcommand{\thefigure}{S\arabic{figure}}% 
}
     
% Variables

\newcommand{\titletext}{Terrestrial laser scanning}

\begin{document}

{\Large{Title: \titletext{}}}


\section*{Abstract}

\section{Introduction}

Assessments of woodland canopy structure are most often conducted using indirect
techniques limited to a two-dimensional perspective. Terrestrial Laser Scanning
(TLS) technology offers a means of assessing canopy structure in three
dimensions. This text should hard wrap at 80 columns, wll does it?

 canopy space filling increases with tree species richness

The characterization of forest structure is a long-standing (Watt, 1947; Whittaker & Woodwell, 1969) research area fundamental to interpreting, modelling, and improving the understanding of eco- system functions. 

Previous studies have attempted to identify the relative contribution of niche complementarity, selection effects, and facilitation effects \citep{Scherer-Lorenzen2014}. These studies have found wide variation in the strength and direction of the BEFR depending on experimental design, the ecosystem being studied, and the environmental conditions \citep{Vila2005}. A focus on functional diversity in BEFR research allows intra-species and inter-species functional variation to be captured and provides reasons why specific species assemblages may exhibit niche complementarity and facilitation effects, while others do not. It also provides a more mechanistic basis for assessing the effect of environmental variation on the BEFR, by exploring how species' functional traits vary across environmental gradients and therefore how their contribution to ecosystem functionality also varies. 

One possible mechanism explaining the observed niche complementarity effect of tree species richness and woody biomass in woodland ecosystems is the diversity of different species' crown architectures. \citet{Solbrig1996} suggest that while trees in SAWs are relatively species poor compared to other biomes, they maintain a high structural/functional diversity. \textit{I.e.} functional redundancy is low. A high diversity of crown shapes at the plot level may allow more efficient use of available light through niche complementarity effects, and possibly direct facilitation by providing low light environments for water limited saplings to prevent drought induced damage such as cavitation, and damage from high temperatures on leaf surfaces. It is unknown however, whether this effect exists in southern African woodlands. Crown packing dynamics in SAWs have been understudied, possibly due to the assumption that adult trees don't interact due to low tree density. Miombo woodlands however often feature relatively high stem densities with tree canopies that clearly interact \citep{Campbell2002}.

As well as horizontal crown packing dynamics, separation of the canopy into distinct vertical layers may also affect plot level woody biomass. In forests, different species may occupy different canopy layers as adult trees, either as an acclimatory/adaptational response to avoid competition with other species, or from being coppiced, via various methods of disturbance \citep{Syampungani2017}. The loss of large trees due to selective logging or reduced competition between trees in low density plots in SAWs may result in forests with simpler vertical structure, which may reduce potential biomass stocks \citep{Mograbi2015}. Disturbance regime may also affect canopy stratification. If fire intensity is low and fires are frequent, the upper canopy may consist mainly of similarly sized old trees that have escaped mortality by fire. In wet tropical forests and temperate forests this vertical canopy differentiation has been well studied \citep{Lowman2004} \citep{Trofymow1998}, but less so in SAWs, perhaps due to the perception that the canopy profile of the woodlands is simplistic. It seems clear however, from observation of woodlands in Angola and Mozambique that vertical canopy stratification does occur, though this remains untested. Individuals of the same species at different growth stages also occupy different canopy layers, with potential for intra-species competition or facilitation effects that may affect woody biomass stocks.

Tree height distribution affects the presence of distinct canopy layers. In SAWs tree size distribution is affected primarily by the degree of disturbance. In the absence of disturbance, a negative exponential tree size distribution naturally occurs due to age dependent mortality \citep{Lehmann2009, Rubin2006}, with departures from that distribution indicating variation in age dependent mortality and recruitment rates over time. Tree size distribution affects spatial heterogeneity of the light environment and distribution of leaf area through the canopy profile \citep{Stark2015}, which in turn is predicted to affect niche differentiation and resource use efficiency. 

Many studies of the biodiversity-ecosystem function relationship in woodlands/forests have focussed on woody biomass stocks and net primary productivity of woody species, while the biomass dynamics of the herbaceous understorey have not received the same level of attention. This is despite understorey strata in woodlands holding the majority of the species richness in a plot \citep{Frost1996}, and performing important ecosystem functions such as regulating forest soil carbon and nutrient pools \citep{Gilliam2007, Nilsson2005}. Understorey species also provide important ecosystem services such as providing grazing fodder for herbivores, provision of thatching grass as a building material, and provision of rare forbs for medicinal purposes \citep{Ryan2016c}. While the biomass held in understorey herbaceous plants may not be significant as a percentage of the total plot biomass, it is important as the principal fuel source for fires, which heavily structure the woodland landscape. Fire promotes further grass growth by maintaining open woodlands via increased mortality among tree saplings \citep{Higgins2007, Hoffmann2012}. 

\citet{Shirima2015a} conducted a small study in Tanzania, comparing Miombo woodland and moist forest plots to investigate the role of canopy structure (measured via Leaf Area Index (LAI) and maximum tree height) on understorey conditions and above-ground herbaceous biomass. They found that higher LAI in Miombo woodlands did not lead to significant variation in herbaceous biomass, while maximum tree height positively correlated with herbaceous biomass. They suggest that canopy cover ameliorates harsh environmental conditions, allowing herbaceous growth. However, they also maintain that the relationships between environmental factors, canopy structure and herbaceous biomass are complex and should be investigated further in other woodlands in eastern and southern Africa.

Understanding mechanisms which trigger tipping points leading to positive feedbacks between fire and grass growth will contribute to the existing debate on alternative stable states in savanna systems. \citet{Hirota2011} argued that there are climatic determinants of tree cover which influence grass growth and therefore disturbance regime through fire, the most influential being mean annual precipitation \citep{Sankaran2005}. I suggest that some of the un-explained variation surrounding the effect of climate on tree cover and therefore grass biomass can be explained by the functional diversity of tree species in the woodland, the diversity of crown shapes and canopy positions that different species occupy. There may also be interactions between the effect of canopy cover on herbaceous biomass and precipitation. In arid, warm savannas, herbaceous growth under trees may benefit from a micro-climate effect which maintains humidity and soil moisture, while in wet savannas this facilitation is unlikely to occur \citep{Dohn2013}. 

I aim to quantify canopy architectural variation across a number of plots in the SEOSAW network to investigate whether greater complementarity in crown packing and greater canopy complexity, in both the vertical and horizontal profile, leads to greater biomass stocks. I will assess whether measures of tree structural diversity improve BEFR models in SAWs. I will also investigate how woody biomass is stratified vertically according to canopy layer and tree position in the canopy. I will include measures of environmental variation between sites to understand how canopy architecture is affected by these variables both within and between species and how this affects woody biomass. 



\section{Materials and methods}

\subsection{Study site}

\subsection{Field measurements}

\subsection{Terrestrial laser scanning}

We used a Leica HDS6100 Terrestrial Laser Scanner, with an angular resolution of
x. This produces a point density of approximately x at x m. Wave length of x nm.

Phase shift

Point clouds within each subplot were registered and unified.

We validated our measurements with traditional digital hemispherical photography
(DHP), with one image taken at the centre point of each subplot.

Acquired our images at the maximum foliage volume in the wet season.

\subsection{Data analysis}

Registration of multiple scans from different locations allows us to minimise
the occlusion effect and improve canopy penetration.

\subsubsection{Scan processing}

Open source methods

Filtering

Voxelisation: Points don't have an area or defined volume and so are unsuitable
for estimating hemispherical points \citep{Seidel2012}.

Objects located closer to the instrument will be represented by a higher density
of points, resulting in an imbalanced representation of the measured 3D space.

Resultant points clouds had ~points, ~points after filtering, ~voxels after
voxelisation.

Classified ground points using the Progressive Morphological Filter (PMF) from
\citep{Zhang2003}. Reclassified height based on this revised ground layer by
measuring the vertical distance between the nearest ground point and each point.

We used ray-tracing (POV-ray) to calculate gap fraction from TLS scans at the
centre of each subplot. Voxels were converted to cubes filling the voxel volume,
with a ``camera'' placed at the subplot centre at 1.8 m height, at a height of
1.8 m. Used a fisheye lens with a view angle of 180 degrees, with matt black
Cubes against a white background and no light source. The images produced by
POV-ray were analysed using Hemiphot in an identical manner to the hemispherical
photographs.

\section{Results}

\section{Discussion}

\section{Conclusion}

\printbibliography

\section{Supplementary Material} \beginsupplement

\end{document}
