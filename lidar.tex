\documentclass[11pt,a4paper]{article}

% Define page geometry
\usepackage{geometry}
\geometry{left=2.2cm,
	right=2.2cm,
	top=2.2cm,
	bottom=2cm}
\parskip 0.15cm
\setlength{\parindent}{0cm}
\usepackage{pdflscape}
\usepackage[document]{ragged2e}

% Text formatting
\usepackage[T1]{fontenc}  % Set font

\usepackage{lineno}  % Line numbers

\usepackage{amssymb}  % Symbols

\linespread{1.25}  % Linespacing

\usepackage{xcolor}
\newcommand{\todo}[1]{\textcolor{red}{\textbf{#1}}}   % todonote

% Tables
\usepackage{multirow}
\setlength{\tabcolsep}{4pt}

% Image handling
\usepackage{graphicx} 

\makeatletter
	\g@addto@macro\@floatboxreset\centering  % Automatically centre images (floats)
\makeatother

\graphicspath{ {img/} }  % Define image path

\usepackage{subfig}  % Compound figures

\usepackage{float}  % Precise figure location

% Bibliography management
\usepackage[style=authoryear, natbib=true, backend=biber]{biblatex}
\addbibresource{lidar.bib}

% Links within document, nice figure formatting
\usepackage[breaklinks]{hyperref}
\definecolor{links}{RGB}{0,0,0}
\hypersetup{
	breaklinks,
	colorlinks=true,
	linkcolor=links,
	anchorcolor=links,
	citecolor=links,
	filecolor=links,
	menucolor=links,
	runcolor=links,
	urlcolor=links,
	pdfauthor={John L. Godlee}
}
\def\subsectionautorefname{section}
\def\subsubsectionautorefname{section}

\newcommand{\beginsupplement}{%
	\setcounter{table}{0}
	\renewcommand{\thetable}{S\arabic{table}}%
	\setcounter{figure}{0}
	\renewcommand{\thefigure}{S\arabic{figure}}%
	}
     
% Variables

\newcommand{\titletext}{Terrestrial laser scanning}

\begin{document}

{\Large{Title: \titletext{}}}


\section*{Abstract}

\section{Introduction}

Assessments of woodland canopy structure are most often conducted using indirect techniques limited to a two-dimensional perspective. Terrestrial Laser Scanning (TLS) technology offers a means of assessing canopy structure in three dimensions.

\section{Materials and methods}

\subsection{Study site}

\subsection{Field measurements}

\subsection{Terrestrial laser scanning}

We used a Leica HDS6100 Terrestrial Laser Scanner, with an angular resolution of x. This produces a point density of approximately x at x m. Wave length of x nm.

Phase shift

Point clouds within each subplot were registered and unified.

We validated our measurements with traditional digital hemispherical photography (DHP), with one image taken at the centre point of each subplot.

Acquired our images at the maximum foliage volume in the wet season.

\subsection{Data analysis}

Registration of multiple scans from different locations allows us to minimise the occlusion effect and improve canopy penetration.

\subsubsection{Scan processing}

Filtering

Voxelisation: Points don't have an area or defined volume and so are unsuitable for estimating hemispherical points \citep{Seidel2012}.

Objects located closer to the instrument will be represented by a higher density of points, resulting in an imbalanced representation of the measured 3D space.

Resultant points clouds had ~points, ~points after filtering, ~voxels after voxelisation.

\section{Results}

\section{Discussion}

\section{Conclusion}

\printbibliography

\section{Supplementary Material}
\beginsupplement

\end{document}
