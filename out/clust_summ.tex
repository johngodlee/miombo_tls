% latex table generated in R 4.1.0 by xtable 1.8-4 package
% Fri Aug  6 13:17:22 2021
\begin{table}
\centering
\caption{Climatic information and Dufrene-Legendre indicator species analysis for the vegetation type clusters identified by the PAM algorithm, based on basal area weighted species abundances. The three species per cluster with the highest indicator values are shown along with other key statistics for each cluster. MAP (Mean Annual Precipitation) and $\delta$T (Diurnal temperature range) are reported as the mean and 1 standard deviation in parentheses. Species richness is reported as the median and the interquartile range in parentheses.} 
\label{clust_summ}
\begin{tabular}{S[table-format=1.0]S[table-format=2.0]rrrrS[table-format=1.2]}
  \toprule
{Cluster} & {N sites} & {Richness} & {Stem dens.} & {AGB} & {Species} & {Ind. value} \\ 
  \midrule
{\multirow{3}{*}{1}} & {\multirow{3}{*}{12}} & {\multirow{3}{*}{17(2)}} & {\multirow{3}{*}{642(194)}} & {\multirow{3}{*}{41( 8.4)}} & Strychnos spinosa & 0.83 \\ 
   & & & & & Combretum collinum & 0.74 \\ 
   & & & & & Julbernardia paniculata & 0.70 \\ 
   \midrule
{\multirow{3}{*}{2}} & {\multirow{3}{*}{5}} & {\multirow{3}{*}{23(4)}} & {\multirow{3}{*}{411(137)}} & {\multirow{3}{*}{72(11.9)}} & Pteleopsis myrtifolia & 1.00 \\ 
  	& & & & &  Diplorhynchus condylocarpon & 0.89 \\ 
  	& & & & & Pseudolachnostylis maprouneifolia & 0.81 \\ 
   \midrule
{\multirow{3}{*}{3}} & {\multirow{3}{*}{3}} &  {\multirow{3}{*}{6(1)}} & {\multirow{3}{*}{196( 55)}} & {\multirow{3}{*}{77( 7.3)}} & Baikiaea plurijuga & 0.94 \\ 
  	& & & &  & Baphia massaiensis & 0.83 \\ 
  	& & & &  & Philenoptera nelsii & 0.45 \\ 
   \midrule
{\multirow{3}{*}{4}} & {\multirow{3}{*}{2}} & {\multirow{3}{*}{12(2)}} & {\multirow{3}{*}{288( 73)}} &  {\multirow{3}{*}{9( 0.2)}} & Vachellia nilotica & 0.99 \\ 
  	& & & & & Combretum apiculatum & 0.70 \\ 
  	& & & & & Senegalia polyacantha & 0.62 \\ 
   \bottomrule
\end{tabular}
\end{table}

