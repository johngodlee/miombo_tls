% latex table generated in R 4.1.0 by xtable 1.8-4 package
% Fri Aug  6 13:17:22 2021
\begin{table}
\centering
\caption{Description of the vegetation type clusters identified using the Ward algorithm, based on basal area weighted genus abundances. Species are indicator species from the Dufr\^{e}ne-Legendre indicator species analysis, with the three highest indicator values. AGB = Above-Ground woody Biomass. Species richness, stem density and AGB are reported as the median and the interquartile range in parentheses.} 
\label{clust_summ}
\setlength\tabcolsep{2pt}
\begin{tabular}{cccccrc}
  \toprule
  {Cluster} & {N plots} & {Richness} & {\makecell{Stem density\\(stems ha\textsuperscript{-1})}} & {\makecell{AGB\\(t ha\textsuperscript{-1})}} & {Indicator species} & {Ind. value} \\ 
  \midrule
{\multirow{3}{*}{1}} & {\multirow{3}{*}{12}} & {\multirow{3}{*}{17(2)}} & {\multirow{3}{*}{642(194)}} & {\multirow{3}{*}{41( 8.4)}} & Strychnos spinosa & 0.83 \\ 
   & & & & & Combretum collinum & 0.74 \\ 
   & & & & & Julbernardia paniculata & 0.70 \\ 
   \midrule
{\multirow{3}{*}{2}} & {\multirow{3}{*}{5}} & {\multirow{3}{*}{23(4)}} & {\multirow{3}{*}{411(137)}} & {\multirow{3}{*}{72(11.9)}} & Pteleopsis myrtifolia & 1.00 \\ 
  	& & & & &  Diplorhynchus condylocarpon & 0.89 \\ 
  	& & & & & Pseudolachnostylis maprouneifolia & 0.81 \\ 
   \midrule
{\multirow{3}{*}{3}} & {\multirow{3}{*}{3}} &  {\multirow{3}{*}{6(1)}} & {\multirow{3}{*}{196( 55)}} & {\multirow{3}{*}{77( 7.3)}} & Baikiaea plurijuga & 0.94 \\ 
  	& & & &  & Baphia massaiensis & 0.83 \\ 
  	& & & &  & Philenoptera nelsii & 0.45 \\ 
   \midrule
{\multirow{3}{*}{4}} & {\multirow{3}{*}{2}} & {\multirow{3}{*}{12(2)}} & {\multirow{3}{*}{288( 73)}} &  {\multirow{3}{*}{9( 0.2)}} & Vachellia nilotica & 0.99 \\ 
  	& & & & & Combretum apiculatum & 0.70 \\ 
  	& & & & & Senegalia polyacantha & 0.62 \\ 
   \bottomrule
\end{tabular}
\end{table}

