% latex table generated in R 4.1.0 by xtable 1.8-4 package
% Tue Aug 10 12:27:13 2021
\begin{table}[]
\centering
\caption{Floristic description of the vegetation type clusters. Species are indicator species from the Dufr\^{e}ne-Legendre indicator species analysis with the three highest indicator values.} 
\label{indval}
\begin{tabular}{crS[table-format=1.2]}
  \toprule
{Cluster} & {Indicator species} & {Indicator value} \\ 
  \midrule
1 & Strychnos spinosa & 0.83 \\ 
  1 & Combretum collinum & 0.74 \\ 
  1 & Julbernardia paniculata & 0.70 \\ 
   \midrule
2 & Pteleopsis myrtifolia & 1.00 \\ 
  2 & Diplorhynchus condylocarpon & 0.89 \\ 
  2 & Pseudolachnostylis maprouneifolia & 0.81 \\ 
   \midrule
3 & Baikiaea plurijuga & 0.94 \\ 
  3 & Baphia massaiensis & 0.83 \\ 
  3 & Philenoptera nelsii & 0.45 \\ 
   \midrule
4 & Vachellia nilotica & 0.99 \\ 
  4 & Combretum apiculatum & 0.70 \\ 
  4 & Senegalia polyacantha & 0.62 \\ 
   \bottomrule
\end{tabular}
\end{table}

