% latex table generated in R 4.0.2 by xtable 1.8-4 package
% Mon Apr 19 15:00:00 2021
\begin{table}[H]
\centering
\begin{tabular}{rcccc}
  \hline
{Site} & \multicolumn{1}{p{2.5cm}}{\centering MAT \\ (\textdegree{}C)} & \multicolumn{1}{p{2.5cm}}{\centering MAP \\ (mm y\textsuperscript{-1})} & \multicolumn{1}{p{2.5cm}}{\centering Temp. range \\ (\textdegree{}C)} & \multicolumn{1}{p{2.5cm}}{\centering CWD \\ (mm y\textsuperscript{-1})} \\
  \hline
Bicuar & 20.8 (0.70) & 825.9 (52.01) & 24.5 (0.90) & -844.8 (44.29) \\ 
  Mtarure & 25.7 (0.24) & 958.4 (25.19) & 12.0 (0.33) & -739.6 (8.06) \\ 
   \hline
\end{tabular}
\caption{Climatic data for each site, extracted from WorldClim at 2.5 minute resolution. Values are the mean and standard deviation (in brackets) of all pixels intersecting each protected area. MAT = Mean Annual Temperature. MAP = Mean Annual Precipitation. Temp. range = Temperature range, calculated as the mean of annual difference between highest temperature of hottest month and lowest temperature of coldest month. CWD = Climatic Water Deficit, calculated as the sum of the difference between monthly rainfall and monthly evapotranspiration when the difference is negative, sensu \citet{Chave2014}.} 
\label{clim}
\end{table}

